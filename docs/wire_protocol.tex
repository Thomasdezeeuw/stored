\documentclass{scrartcl}

\title{Write protocol description}
\author{Thomas de Zeeuw}
\date{June 16th 2019}

\begin{document}

\maketitle

\section{WireProtocol} \label{sec:wire_protocol}

TODO: expand on this.

Request has the following structure:
* request type: 1 byte.
* request data: n bytes, format and length depends on the request type,

Request types:
* retrieve,
* store,
* remove.

retrieve has a `value' as data.
Store and remove have `key' as data.

Value has the following structure:
* value type: 1 byte, see below.
* value data: n byte, depends on the value type.

Value type can be one of the following:
* small blob, length as u16, the bytes that make the value.
* blob, length as u32, the bytes that make the value.
* large blob, length as u64, the bytes that make the value.

More value type can be added in the future. For example an JSON type has only
accepts valid JSON. Or maybe even allow for use defined schema's that valid the
data.

Key is always the bytes that make up the sha512 checksum.

\end{document}
