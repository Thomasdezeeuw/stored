\documentclass{scrartcl}

\title{Server architecture}
\author{Thomas de Zeeuw}
\date{June 16th 2019}

\begin{document}

\maketitle

\section{Server architecture} \label{sec:srv}

The Coeus server consists of a number of components, each with there own task
and responsibility. Figure \ref{fig:server_design} gives an overview of how all
the components interact. The following sections describe each component and
the interactions between them.


\begin{figure}
\begin{verbatim}
 +---------+               +----------------+
 |         |     Updates   |                |
 |  Cache  <---------------+  Cache master  |   // Single synchronous
 |         |               |                |   // Heph actor.
 +----^----+               +--------^-------+
      |                             |
Reads |                             | Update
 from |                             |  (msg)
      |              +----------+   |
      |           +-----------+ |   |
      |         +-----------+ | +---+
      |         |           | | |               // Regular (local)
      +---------+  Clients  | +-+               // Heph actors.
                |           +-+
                +----+------+
                     |
                     | Sends I/O operations
                     |
                +----|---------+
              +------v-------+ |
              |              | |
              |  I/O thread  | |
              |     pool     | |
              |              +-+
              +------+-------+
                     |
                     | Synchronously reads to/writes from
                     |
                 +---v----+
                 |        |
                 |  Disk  |
                 |        |
                 +--------+
\end{verbatim}
\caption{Architecture for the server.}
\label{fig:server_design}
\end{figure}


\subsection{Client} \label{sec:srv_client}

The client is an actor that deals with incoming connections. It's a regular
(local) Heph actor for which one is created per incoming connection. It will
parse any requests written to the connection and will handle them.

It will have read only access to the cache. It has an actor reference to the
cache master, which is used to add and remove value from the cache. For disk I/O
it makes use of the I/O thread pool, to which it sends operations to execute
(e.g. storing values).


\subsection{Cache master} \label{sec:srv_cache_master}

The cache master is the only actor with write access to the cache. It is a
synchronous Heph actor that accept messages from the clients, adding or removing
values.


\subsection{Cache} \label{sec:srv_cache}

Conceptually cache is a map: key $ \rightarrow $ value. It allows multiple
readers, the clients, but only a single writer, the cache master. Currently this
is implemented by using the `evmap' crate. The values are reference counted to
ensure values always live long enough.


\subsection{I/O pool} \label{sec:srv_io_pool}

The I/O consists of a number of threads that execute synchronous disk I/O
operations. As doing asynchronous disk I/O is still unpleasant the operations
are moved to separate threads that deal with the operations.

Operations can be request by the clients and are scheduled on one of the
available threads, which synchronously execute the operation on the disk.


\end{document}
